\section{Power Cycles}
Much of the power that we use in our daily lives is produced through cycles operating with a working fluid.

%%%%%%%%%%%%%%%%%%%%%%%%%%%%%%%%%%%%%%%%%%%%%%%%%%%%%%%%%%%%%%%%%%%%%%%%%%%%%%%%%%%%%%%%%%%%%%%%%%%%%%%%%%%%%%%%%%%%%%%%%%%%%%%%%%%%
\subsection{Carnot cycle}
A \textbf{Carnot cycle} is defined as an ideal gas, undergoing a 4-step, reversible, closed thermodynamic cycle. Imagine that an ideal gas contained within a piston-cylinder undergoes the series of processes depicted in Fig.~\ref{fig:CarnotCycle} below, with the gas starting in state $1$ before progressing to states $2$, $3$, $4$, and then returning to state $1$ to complete a cycle. 
%%%%%%%%%%%%%%%%%%%%%%%%%%%%%%%%%%%%%%%%%%%%%%%%%%%%%%%%%%%%%%%%%%%%%%%%%%%%%%%%%%%%%%%%%%%%%%%%%%%%%%%%%%%%%%%%%%
\begin{figure}[h]
\begin{center}
\begin{tikzpicture}[auto,every text node part/.style={align=center},>=latex']
%pistons
% State1 cylinder
	\node (ref) at (0,0) [draw=none, coordinate] {};
	\draw [ thick, join=round] ($ (ref) + (2,1)$) --  ($ (ref) + (0,1) $) -- (ref) -- ($ (ref) + (2,0)$);
	\draw (ref) [ thick, fill=Mteal] rectangle($ (ref) + (1,1)$); % define rectangle by lower left and upper right points
	\draw ($ (ref) + (1,0)$) [ thick, fill=gray!80] rectangle($ (ref) + (1.2,1)$); % define rectangle by lower left and upper right points
	\draw ($ (ref) + (0,1.9)$) node[circle, inner sep=0.5pt, draw]  {\footnotesize $1$};
	\draw [ultra thick, -latex] ($ (ref) + (1.4,0.5)$)  node[right, xshift=-0.15cm] {$|W_{12}|$} -- ($ (ref) + (0.5,0.5)$);
	\draw [ draw=none, pattern=north west lines, pattern color=black] (ref) -- ($ (ref) + (0,1)$) -- ($ (ref) + (2,1)$) -- ($ (ref) + (2,1.25)$) -- ($ (ref) + (-0.25,1.25)$) -- ($ (ref) + (-0.25,-0.25)$) -- ($ (ref) + (2,-0.25)$) -- ($ (ref) + (2,0)$) -- cycle; % insulated region
	\draw ($ (ref) + (-0.25,1.15)$) [ draw= none, fill=red!70] rectangle($ (ref) + (2,1.5)$); % define rectangle by lower left and upper right points, hot reservoir
	\draw ($ (ref) + (-0.25,1.45)$) [ draw= none, top color=white, bottom color=red!70] rectangle($ (ref) + (2,1.7)$); % define rectangle by lower left and upper right points, hot reservoir
	\draw ($ (ref) + (-0.25,-0.5)$) [ draw= none, fill=Dlblue] rectangle($ (ref) + (2,-0.15)$); % define rectangle by lower left and upper right points, cold reservoir
	\draw ($ (ref) + (-0.25,-0.45)$) [ draw= none, top color=Dlblue, bottom color=white] rectangle($ (ref) + (2,-0.7)$); % define rectangle by lower left and upper right points, hot reservoir
	\draw ($ (ref) + (0.875,-1)$) node[draw = none]  {\footnotesize $1 \to 2$};

% State2 cylinder
	\node (ref2) at  ($ (ref) + (2.5,0)$)  [draw=none, coordinate] {};
	\draw (ref2) [ thick, fill=Mteal] rectangle($ (ref2) + (0.5,1)$); % define rectangle by lower left and upper right points
	\draw ($ (ref2) + (0.5,0)$) [ thick, fill=gray!80] rectangle($ (ref2) + (0.7,1)$); % define rectangle by lower left and upper right points
	\draw ($ (ref2) + (0,1.9)$) node[circle, inner sep=0.5pt, draw]  {\footnotesize $2$};
	\draw [ultra thick, latex-] ($ (ref2) + (1.1,0.5)$)  node[right, xshift=-0.15cm] {$|W_{23}|$} -- ($ (ref2) + (0.25,0.5)$);
	\draw [ draw=none, pattern=north west lines, pattern color=black] (ref2) -- ($ (ref2) + (0,1)$) -- ($ (ref2) + (2,1)$) -- ($ (ref2) + (2,1.25)$) -- ($ (ref2) + (-0.25,1.25)$) -- ($ (ref2) + (-0.25,-0.25)$) -- ($ (ref2) + (2,-0.25)$) -- ($ (ref2) + (2,0)$) -- cycle; % insulated region
	\draw ($ (ref2) + (-0.25,1)$) [ draw= none, fill=red!70] rectangle($ (ref2) + (2,1.5)$); % define rectangle by lower left and upper right points, hot reservoir
	\draw ($ (ref2) + (-0.25,1.45)$) [ draw= none, top color=white, bottom color=red!70] rectangle($ (ref2) + (2,1.7)$); % define rectangle by lower left and upper right points, hot reservoir
	\draw ($ (ref2) + (-0.25,-0.5)$) [ draw= none, fill=Dlblue] rectangle($ (ref2) + (2,-0.15)$); % define rectangle by lower left and upper right points, cold reservoir
	\draw ($ (ref2) + (-0.25,-0.45)$) [ draw= none, top color=Dlblue, bottom color=white] rectangle($ (ref2) + (2,-0.7)$); % define rectangle by lower left and upper right points, hot reservoir
	\draw ($ (ref2) + (0.875,-1)$) node[draw = none]  {\footnotesize $2 \to 3$};
	\draw [ thick, join=round] ($ (ref2) + (2,1)$) --  ($ (ref2) + (0,1) $) -- (ref2) -- ($ (ref2) + (2,0)$);
	\draw[-latex,decorate,decoration={snake,amplitude=.5mm,segment length=1.5mm ,post length=6pt}, very thick, draw=red!70, color=red!70] ($ (ref2) + (0.25,1.05)$) -- ($ (ref2) + (0.25,0.55)$)  node[below, text=red!70] {\footnotesize $Q_H$} ;
	\draw ($ (ref2) + (0.875,1.25)$) node[draw = none, text=white]  {\footnotesize hot reservoir $T_H$};
	%\draw ($ (ref2) + (-0.25,1.8)$) node[draw = none, text=red!20]  {\footnotesize $T_H$};
	
% State3 cylinder
	\node (ref3) at  ($ (ref2) + (2.4,0)$)  [draw=none, coordinate] {};
	\draw [ thick, join=round] ($ (ref3) + (2,1)$) --  ($ (ref3) + (0,1) $) -- (ref3) -- ($ (ref3) + (2,0)$);
	\draw (ref3) [ thick, fill=Mteal] rectangle($ (ref3) + (1,1)$); % define rectangle by lower left and upper right points
	\draw ($ (ref3) + (1,0)$) [ thick, fill=gray!80] rectangle($ (ref3) + (1.2,1)$); % define rectangle by lower left and upper right points
	\draw ($ (ref3) + (0,1.9)$) node[circle, inner sep=0.5pt, draw]  {\footnotesize $3$};
	\draw [ultra thick, latex-] ($ (ref3) + (1.6,0.5)$)  node[right,xshift=-0.15cm] {$|W_{34}|$} -- ($ (ref3) + (0.6,0.5)$);
	\draw [ draw=none, pattern=north west lines, pattern color=black] (ref3) -- ($ (ref3) + (0,1)$) -- ($ (ref3) + (2,1)$) -- ($ (ref3) + (2,1.25)$) -- ($ (ref3) + (-0.25,1.25)$) -- ($ (ref3) + (-0.25,-0.25)$) -- ($ (ref3) + (2,-0.25)$) -- ($ (ref3) + (2,0)$) -- cycle; % insulated region
	\draw ($ (ref3) + (-0.25,1.15)$) [ draw= none, fill=red!70] rectangle($ (ref3) + (2,1.5)$); % define rectangle by lower left and upper right points, hot reservoir
	\draw ($ (ref3) + (-0.25,1.45)$) [ draw= none, top color=white, bottom color=red!70] rectangle($ (ref3) + (2,1.7)$); % define rectangle by lower left and upper right points, hot reservoir
	\draw ($ (ref3) + (-0.25,-0.5)$) [ draw= none, fill=Dlblue] rectangle($ (ref3) + (2,-0.15)$); % define rectangle by lower left and upper right points, cold reservoir
	\draw ($ (ref3) + (-0.25,-0.45)$) [ draw= none, top color=Dlblue, bottom color=white] rectangle($ (ref3) + (2,-0.7)$); % define rectangle by lower left and upper right points, hot reservoir
	\draw ($ (ref3) + (0.875,-1)$) node[draw = none]  {\footnotesize $3 \to 4$};
	
% State4 cylinder
	\node (ref4) at  ($ (ref3) + (2.7,0)$)  [draw=none, coordinate] {};
	\draw (ref4) [ thick, fill=Mteal] rectangle($ (ref4) + (1.5,1)$); % define rectangle by lower left and upper right points
	\draw ($ (ref4) + (1.5,0)$) [ thick, fill=gray!80] rectangle($ (ref4) + (1.7,1)$); % define rectangle by lower left and upper right points
	\draw ($ (ref4) + (0,1.9)$) node[circle, inner sep=0.5pt, draw]  {\footnotesize $4$};
	\draw [ultra thick, -latex] ($ (ref4) + (1.9,0.5)$)  node[right,xshift=-0.15cm] {$|W_{41}|$} -- ($ (ref4) + (1.1,0.5)$);
	\draw [ draw=none, pattern=north west lines, pattern color=black] (ref4) -- ($ (ref4) + (0,1)$) -- ($ (ref4) + (2,1)$) -- ($ (ref4) + (2,1.25)$) -- ($ (ref4) + (-0.25,1.25)$) -- ($ (ref4) + (-0.25,-0.25)$) -- ($ (ref4) + (2,-0.25)$) -- ($ (ref4) + (2,0)$) -- cycle; % insulated region
	\draw ($ (ref4) + (-0.25,1.15)$) [ draw= none, fill=red!70] rectangle($ (ref4) + (2,1.5)$); % define rectangle by lower left and upper right points, hot reservoir
	\draw ($ (ref4) + (-0.25,1.45)$) [ draw= none, top color=white, bottom color=red!70] rectangle($ (ref4) + (2,1.7)$); % define rectangle by lower left and upper right points, hot reservoir
	\draw ($ (ref4) + (-0.25,-0.5)$) [ draw= none, fill=Dlblue] rectangle($ (ref4) + (2,0)$); % define rectangle by lower left and upper right points, cold reservoir
	\draw ($ (ref4) + (-0.25,-0.45)$) [ draw= none, top color=Dlblue, bottom color=white] rectangle($ (ref4) + (2,-0.7)$); % define rectangle by lower left and upper right points, hot reservoir
	\draw ($ (ref4) + (0.875,-1)$) node[draw = none]  {\footnotesize $4 \to 1$};
	\draw [ thick, join=round] ($ (ref4) + (2,1)$) --  ($ (ref4) + (0,1) $) -- (ref4) -- ($ (ref4) + (2,0)$);
	\draw[-latex,decorate,decoration={snake,amplitude=.5mm,segment length=1.5mm ,post length=6pt}, very thick, draw=Dlblue, color=Dlblue] ($ (ref4) + (0.75,0.5)$) node[left, text=Dlblue] {\footnotesize $Q_C$} -- ($ (ref4) + (0.75,-0.05)$);
	\draw ($ (ref4) + (0.875,-0.25)$) node[draw = none, text=white]  {\footnotesize cold reservoir $T_C$};

\end{tikzpicture}\end{center}
\caption{A schematic of a piston-cylinder executing a Carnot cycle between hot and cold thermal reservoirs. }\label{fig:CarnotCycle}
\end{figure}
%%%%%%%%%%%%%%%%%%%%%%%%%%%%%%%%%%%%%%%%%%%%%%%%%%%%%%%%%%%%%%%%%%%%%%%%%%%%%%%%%%%%%%%%%%%%%%%%%%%%%%%%%%%%%%%%%%

During the cycle, the system contacts sequentially a hot reservoir and a cold reservoir at temperatures $T_H$ and $T_C$, respectively. While in contact with a thermal reservoir, the system remains at the temperature of the thermal reservoir. In between reservoir contacts, the system is insulated and no heat transfer occurs. Each process \textbf{quasi-static}, and \textbf{reversible}. In this idealized cycle, heat transfers in and out of the system despite the fact that there is no temperature difference to drive heat flow. Such hypothetical heat flow is reversible because an infinitesimal increase in the reservoir temperature relative to the system will cause heat to flow into the system, or vice versa for an infinitesimal decrease in the reservoir temperature. This idealized cycle is known as a Carnot cycle. 

\paragraph{Reversible process:} In a reversible process, the direction can be reversed at any point by an infinitesimal change in external conditions. 

\paragraph{Quasi-static process:} A process that occurs at an infinitesimally slow rate so that the system is in a thermodynamic equilibrium at all times.

\subsubsection{Carnot Cycle Steps}

In processes $2 \to 3$ and $3 \to 4$, work is done by the gas. Figure~\ref{fig:PVwork1} illustrates two potential, monotonic processes corresponding to positive work, or work done by the gas. On the left, pressure is higher in state $b$ than state $a$. On the right, pressure is higher in state $a$ than state $b$. $V_b > V_a$ in both cases. When illustrating the magnitude of the work to be determined, the arrow is drawn out of the system as shown in state $1$. 


\begin{figure}[h]
\begin{center}
\begin{tikzpicture}[auto,>=latex',
	declare function={
        		curve1(\x) = \x^2;
		curve2(\x) = 1/(\x^2);
  	}
	]

% upper cylinder
	\node (ref) at (0,0) [draw=none, coordinate] {};
	\draw [ thick, join=round] ($ (ref) + (2,1)$) -- ($ (ref) + (0,1) $) -- (ref) -- ($ (ref) + (2,0)$);
	\draw (ref) [ thick, fill=Dlblue] rectangle($ (ref) + (0.75,1)$); % define rectangle by lower left and upper right points
	\draw ($ (ref) + (0.75,0)$) [ thick, fill=gray!80] rectangle($ (ref) + (0.95,1)$); % define rectangle by lower left and upper right points
	\draw ($ (ref) + (-0.25,1)$) node[circle, inner sep=0.5pt, draw]  {\footnotesize $a$};
	\draw [ultra thick, -latex] ($ (ref) + (0.5,0.5)$)  -- ($ (ref) + (1.5,0.5)$) node[above] {$W$};
% lower cylinder
	\node (ref2) at ($ (ref) - (0,2)$) [draw=none, coordinate] {};
	\draw [ thick, join=round] ($ (ref2) + (2,1)$) -- ($ (ref2) + (0,1) $) -- (ref2) -- ($ (ref2) + (2,0)$);
	\draw (ref2) [ thick, fill=Mteal] rectangle($ (ref2) + (1.3,1)$); % define rectangle by lower left and upper right points
	\draw ($ (ref2) + (1.3,0)$) [ thick, fill=gray!80] rectangle($ (ref2) + (1.5,1)$);  % define rectangle by lower left and upper right points
	\draw ($ (ref2) + (-0.25,1)$) node[circle, inner sep=0.5pt, draw]  {\footnotesize $b$};
	% arrow between
	\node [isosceles triangle, draw, fill=black, isosceles triangle stretches, minimum height=0.2cm, minimum width=0.5cm, shape border rotate=-90, inner sep=0pt] at ($ (ref) + (1,-0.5)$) {};
	
% left plot
	\node (refp1) at ($ (ref) + (2.75,-2)$) [draw=none, coordinate] {};
	\begin{axis} [at={(refp1)},xmin=1, ymin=0.5, xmax=2, ymax = 4.2, samples = 50, 
			ytick=\empty, yticklabels={}, ylabel style={rotate=-90},
			xtick=\empty, xticklabels={},
			ylabel={$p$}, xlabel={$V$},
			width = 4.5cm, height=4.5cm, axis y line = left, axis x line = bottom,
			domain = 1.1:1.9, fill between/on layer={axis background}]
		
		\addplot[name path=f, black, very thick,->] {curve1(x)}; % Curve
		\path[name path=ax] (axis cs:1, 0.5) -- (axis cs:2, 0.5); % Axis base to shade to
		\addplot [left color = Dlblue, right color = Mteal] % Add shaded region
		fill between [
			of = f and ax,
			soft clip={domain=1.1:1.9},
		];
		\addplot [only marks, very thick] coordinates { % Add end points
		(1.1, {curve1(1.1)})
		(1.9, {curve1(1.9)})
		};
		\node [above, color=black] at (axis cs:1.1, {curve1(1.1)}) {$a$};
		\node [above, color=black] at (axis cs:1.9, {curve1(1.9)}) {$b$};
		
	\end{axis}
	\draw ($ (refp1) + (1.5,2.8)$) node[draw=none] {$p_b>p_a$};

% right plot
	\node (refp2) at ($ (refp1) + (3.5,0)$) [draw=none, coordinate] {};
	\begin{axis} [at={(refp2)},xmin=1, ymin=0.1, xmax=2, ymax = 1, samples = 50, 
			ytick=\empty, yticklabels={}, ylabel style={rotate=-90},
			xtick=\empty, xticklabels={},
			ylabel={$p$}, xlabel={$V$},
			width = 4.5cm, height=4.5cm, axis y line = left, axis x line = bottom,
			domain = 1.1:1.9, fill between/on layer={axis background}]
		
		\addplot[name path=g, black, very thick,->] {curve2(x)};
		\path[name path=ax2] (axis cs:1, 0.1) -- (axis cs:2, 0.1);
		\addplot [left color = Dlblue, right color = Mteal]
		fill between [
			of = g and ax2,
			soft clip={domain=1.1:1.9},
		];
		\addplot [only marks, very thick] coordinates {
		(1.1, {curve2(1.1)})
		(1.9, {curve2(1.9)})
		};
		\node [above, color=black] at (axis cs:1.1, {curve2(1.1)}) {$a$};
		\node [above, color=black] at (axis cs:1.9, {curve2(1.9)}) {$b$};
	\end{axis}
	\draw ($ (refp2) + (1.5,2.8)$) node[draw=none] {$p_a>p_b$};

\end{tikzpicture}
\end{center}
\caption{Work done by the gas}\label{fig:PVwork1}
\end{figure}


In processes $1 \to 2$ and $4 \to 1$, work is done on the gas. Figure~\ref{fig:PVwork1} illustrates two potential, monotonic processes corresponding to positive work, or work done by the gas. On the left, pressure is higher in state $b$ than state $a$. On the right, pressure is higher in state $a$ than state $b$. $V_a > V_b$ in both cases as shown in Figure~\ref{fig:PVwork2}. Recall the expression for $pV$ work,
\begin{equation}
W = \int_{V_a}^{V_b} p(V) dV,
\end{equation}
noting that work for these processes is negative as indicated by the right-to-left process curve arrow in each of the $pV$ diagrams. When illustrating the magnitude of the work to be determined, the arrow is drawn into the system as shown in state $a$ (Fig.~\ref{fig:PVwork2}). 

\begin{figure}[h]
\begin{center}
\begin{tikzpicture}[auto,>=latex',
	declare function={
        		curve1(\x) = 1/(\x^2);
		curve2(\x) = \x^2;
  	}
	]

% upper cylinder
	\node (ref) at (0,0) [draw=none, coordinate] {};
	\draw [ thick, join=round] ($ (ref) + (2,1)$) -- ($ (ref) + (0,1) $) -- (ref) -- ($ (ref) + (2,0)$);
	\draw (ref) [ thick, fill=Mteal] rectangle($ (ref) + (1.3,1)$); % define rectangle by lower left and upper right points
	\draw ($ (ref) + (1.3,0)$) [ thick, fill=gray!80] rectangle($ (ref) + (1.5,1)$);  % define rectangle by lower left and upper right points
	\draw ($ (ref) + (-0.25,1)$) node[circle, inner sep=0.5pt, draw]  {\footnotesize $a$};
	\draw [ultra thick, latex-] ($ (ref) + (0.75,0.5)$)  -- ($ (ref) + (1.75,0.5)$) node[above] {$W$};
% lower cylinder
	\node (ref2) at ($ (ref) - (0,2)$) [draw=none, coordinate] {};
	\draw [ thick, join=round] ($ (ref2) + (2,1)$) -- ($ (ref2) + (0,1) $) -- (ref2) -- ($ (ref2) + (2,0)$);
	\draw (ref2) [ thick, fill=Dlblue] rectangle($ (ref2) + (0.75,1)$); % define rectangle by lower left and upper right points
	\draw ($ (ref2) + (0.75,0)$) [ thick, fill=gray!80] rectangle($ (ref2) + (0.95,1)$); % define rectangle by lower left and upper right points
	\draw ($ (ref2) + (-0.25,1)$) node[circle, inner sep=0.5pt, draw]  {\footnotesize $b$};
	% arrow between
	\node [isosceles triangle, draw, fill=black, isosceles triangle stretches, minimum height=0.2cm, minimum width=0.5cm, shape border rotate=-90, inner sep=0pt] at ($ (ref) + (1,-0.5)$) {};
	
% left plot
	\node (refp1) at ($ (ref) + (2.75,-2)$) [draw=none, coordinate] {};
	\begin{axis} [at={(refp1)},xmin=1, ymin=0.1, xmax=2, ymax = 1, samples = 50, 
			ytick=\empty, yticklabels={}, ylabel style={rotate=-90},
			xtick=\empty, xticklabels={},
			ylabel={$p$}, xlabel={$V$},
			width = 4.5cm, height=4.5cm, axis y line = left, axis x line = bottom,
			domain = 1.1:1.9, fill between/on layer={axis background}]
		
		\addplot[name path=f, black, very thick,<-] {curve1(x)}; % Curve
		\path[name path=ax] (axis cs:1, 0.1) -- (axis cs:2, 0.1); % Axis base to shade to
		\addplot [left color = Dlblue, right color = Mteal] % Add shaded region
		fill between [
			of = f and ax,
			soft clip={domain=1.1:1.9},
		];
		\addplot [only marks, very thick] coordinates { % Add end points
		(1.1, {curve1(1.1)})
		(1.9, {curve1(1.9)})
		};
		\node [above, color=black] at (axis cs:1.1, {curve1(1.1)}) {$b$};
		\node [above, color=black] at (axis cs:1.9, {curve1(1.9)}) {$a$};
		
	\end{axis}
	\draw ($ (refp1) + (1.5,2.8)$) node[draw=none] {$p_b>p_a$};

% right plot
	\node (refp2) at ($ (refp1) + (3.5,0)$) [draw=none, coordinate] {};
	\begin{axis} [at={(refp2)},xmin=1, ymin=0.5, xmax=2, ymax = 4.2, samples = 50, 
			ytick=\empty, yticklabels={}, ylabel style={rotate=-90},
			xtick=\empty, xticklabels={},
			ylabel={$p$}, xlabel={$V$},
			width = 4.5cm, height=4.5cm, axis y line = left, axis x line = bottom,
			domain = 1.1:1.9, fill between/on layer={axis background}]
		
		\addplot[name path=g, black, very thick,<-] {curve2(x)};
		\path[name path=ax2] (axis cs:1, 0.5) -- (axis cs:2, 0.5);
		\addplot [left color = Dlblue, right color = Mteal]
		fill between [
			of = g and ax2,
			soft clip={domain=1.1:1.9},
		];
		\addplot [only marks, very thick] coordinates {
		(1.1, {curve2(1.1)})
		(1.9, {curve2(1.9)})
		};
		\node [above, color=black] at (axis cs:1.1, {curve2(1.1)}) {$b$};
		\node [above, color=black] at (axis cs:1.9, {curve2(1.9)}) {$a$};
	\end{axis}
	\draw ($ (refp2) + (1.5,2.8)$) node[draw=none] {$p_a>p_b$};

\end{tikzpicture}
\end{center}
\caption{Work done on the gas}\label{fig:PVwork2}
\end{figure}

To plot the processes that make up the cycle in Fig.~\ref{fig:CarnotCycle}, we first observe that processes $2 \to 3$ and $4 \to 1$ are \textbf{isothermal}, meaning the temperature of the working fluid is constant during the process. Given that the working fluid is an ideal gas, $pV = mRT$ and therefore 
\begin{equation}\label{eq:IG_eos2}
p = \frac{mR}{V} T
\end{equation}
where $m$ is the mass of gas and $R$ is the gas constant. An isothermal process at $T_H$ or $T_C$, the temperature of the hot and cold reservoirs, must therefore fall on the dashed lines in Fig.~\ref{fig:CarnotpV}. These lines are known as \textbf{isotherms}. 

\begin{figure}[h]
\begin{center}
\begin{tikzpicture}[auto,>=latex',
	declare function={
        		curveTH(\x) = 1/(\x);
		curveTC(\x) = 0.5/(\x);
		curveR(\x) = 0.8/(\x^(2));
		curveL(\x) = 0.4/(\x^(2));
  	}
	]

% plot
	\node (ref) at (0,0) [draw=none, coordinate] {};
	\begin{axis} [at={(ref)}, xmin=0.3, ymin=0.2, xmax=1.8, ymax = 2.6, samples = 100, 
			ytick=\empty, yticklabels={}, ylabel style={rotate=-90},
			xtick=\empty, xticklabels={},
			ylabel={$p$}, xlabel={$V$},
			width = 5cm, height=5cm, axis y line = left, axis x line = bottom,
			domain = 0.35:1.7, fill between/on layer={axis background}]
		
		\addplot[red!70, very thick, dashed] {curveTH(x)};
		\addplot[Dlblue, very thick,dashed] {curveTC(x)};
		\addplot[black!50, very thick,dotted] {curveL(x)};
		\addplot[black, very thick,dotted] {curveR(x)};

	\end{axis}
	\node (refL) at ($ (ref) + (4,3)$) [draw=none, coordinate] {};
	\draw [dashed, very thick, red!70] ($ (refL) + (0,0)$) -- ($ (refL) + (1cm,0)$) node[right, draw=none] {$T_H$};
	\draw [dashed, very thick, Dlblue] ($ (refL) + (0,-0.5cm)$) -- ($ (refL) + (1cm,-0.5cm)$) node[right, draw=none] {$T_C$};
	\draw [dotted, very thick, black!50] ($ (refL) + (0,-1cm)$) -- ($ (refL) + (1cm,-1cm)$) node[right, draw=none] {$C = p_1V_1^{\gamma} = p_2V_2^{\gamma}$};
	\draw [dotted, very thick, black] ($ (refL) + (0,-1.5cm)$) -- ($ (refL) + (1cm,-1.5cm)$) node[right, draw=none] {$C = p_3V_3^{\gamma} = p_4V_4^{\gamma}$};

\end{tikzpicture}
\end{center}
\caption{$pV$ curves for ideal gas expansion and compression in a Carnot cycle.}\label{fig:CarnotpV}
\end{figure}

Processes $1 \to 2$ and $3 \to 4$ are \textbf{adiabatic}, meaning that during the process no heat transfer occurs. We will show later in the class that the adiabatic, reversible expansion or compression of an ideal gas must obey the relation
\begin{equation}\label{eq:polytropic}
pV^{\gamma}=C
\end{equation}
where $\gamma$ and $C$ are constants and $\gamma > 1$. Therefore these processes must fall on a curve defined by
\begin{equation}
\label{eq:Pexpression}
p(V)=\frac{C}{V^{\gamma}},
\end{equation}
illustrated for two values of $C$ by the dash-dot lines in Fig.~\ref{fig:CarnotpV}. 

Combining the $p(V)$ relations for the ideal gas expansion and compression with the directionality of the work shown in Figs.~\ref{fig:PVwork1} and \ref{fig:PVwork2}, we can plot a Carnot cycle for an ideal gas on a $pV$ diagram as shown in Fig.~\ref{fig:CarnotpVcycle}. 

\begin{figure}[h]
\begin{center}
\begin{tikzpicture}[auto,>=latex',
	declare function={
        		curveTH(\x) = 1/(\x);
		curveTC(\x) = 0.5/(\x);
		curveR(\x) = 0.8/(\x^(2));
		curveL(\x) = 0.4/(\x^(2));
  	}
	]

% plot
	\node (ref) at (0,0) [draw=none, coordinate] {};
	\begin{axis} [at={(ref)}, xmin=0.3, ymin=0.2, xmax=1.8, ymax = 2.6, samples = 100, 
			ytick=\empty, yticklabels={}, ylabel style={rotate=-90},
			xtick=\empty, xticklabels={},
			ylabel={$p$}, xlabel={$V$},
			width = 7cm, height=7cm, axis y line = left, axis x line = bottom,
			domain = 0.35:1.7, fill between/on layer={axis background}]
		
		\addplot[red!70,  thick, dashed] {curveTH(x)};
		\addplot[Dlblue,  thick,dashed] {curveTC(x)};
		\addplot[black!50,  thick,dotted] {curveL(x)};
		\addplot[black,  thick,dotted] {curveR(x)};
		\addplot[name path=hot, Ddblue, very thick, domain=0.4:0.8, ->] {curveTH(x)};
		\addplot[name path=cold, Ddblue, very thick, domain={0.4/0.5}:{0.8/0.5}, <-] {curveTC(x)};
		\addplot[name path=left, Ddblue, very thick, domain=0.4:{0.4/0.5}, <-] {curveL(x)};
		\addplot[name path=right, Ddblue, very thick, domain=0.8:{0.8/0.5}, ->] {curveR(x)};
		\addplot [Mteal, draw=Mteal]
		fill between [
			of = hot and left,
			soft clip={domain=0.4:0.8},
		];
		\addplot [Mteal]
		fill between [
			of = right and cold,
			soft clip={domain=0.8:{0.8/0.5}},
		];
		\addplot [only marks, thick] coordinates {
		({0.4/0.5}, {curveTC(0.4/0.5)})
		(0.4, {curveTH(0.4)})
		(0.8, {curveTH(0.8)})
		({0.8/0.5}, {curveTC(0.8/0.5)})
		};
		\node [below, color=black] at (axis cs:{0.4/0.5}, {curveTC(0.4/0.5)}) {$1$};
		\node [right, color=black] at (axis cs:{0.4}, {curveTH(0.4)}) {$2$};
		\node [right, color=black] at (axis cs:{0.8}, {curveTH(0.8)}) {$3$};		
		\node [above, color=black] at (axis cs:{0.8/0.5}, {curveTC(0.8/0.5)}) {$4$};
	\end{axis}
	\node (refL) at ($ (ref) + (3,5)$) [draw=none, coordinate] {};
	\draw [dashed,  thick, red!70] ($ (refL) + (0,0)$) -- ($ (refL) + (1cm,0)$) node[right, draw=none] {$T_H$};
	\draw [dashed,  thick, Dlblue] ($ (refL) + (0,-0.5cm)$) -- ($ (refL) + (1cm,-0.5cm)$) node[right, draw=none] {$T_C$};
	\draw [dotted,  thick, black!50] ($ (refL) + (0,-1cm)$) -- ($ (refL) + (1cm,-1cm)$) node[right, draw=none] {$C_{12} = p_1V_1^{\gamma} = p_2V_2^{\gamma}$};
	\draw [dotted,  thick, black] ($ (refL) + (0,-1.5cm)$) -- ($ (refL) + (1cm,-1.5cm)$) node[right, draw=none] {$C_{34} = p_3V_3^{\gamma} = p_4V_4^{\gamma}$};

\end{tikzpicture}
\end{center}
\caption{A $pV$ diagram for an ideal gas undergoing a Carnot cycle as depicted in Fig.~\ref{fig:CarnotCycle}.}\label{fig:CarnotpVcycle}
\end{figure}

The work done by the cycle $W$ is positive and equal to the shaded area in Fig.~\ref{fig:CarnotpVcycle}, where the net work is described as the sum of the work done during each of the processes,
\begin{equation}
W= W_{12} + W_{23} + W_{34} + W_{41}
\end{equation}

\subsubsection{Isothermal Processes}
For isothermal processes $2 \to 3$ and $4 \to 1$ the work is determined by the $p(V)$ relation in Eqn~\eqref{eq:IG_eos2}
which yields
\begin{equation}
W_{23} =mRT_H\ln\left(\frac{V_3}{V_2}\right) \,\,\text{ and }\,\, W_{41} = mRT_C\ln\left(\frac{V_1}{V_4}\right)
\end{equation}
Because temperature is not changing and the fluid is an ideal gas, the change in internal energy for these processes is zero. An energy balance on the gas for each of these processes yields
\begin{equation}\label{eq:QH}
Q_{23} = Q_{H} = W_{23} = mRT_H\ln\left(\frac{V_3}{V_2}\right) 
\end{equation}
where $Q_H$ refers to the heat transferred into the system from the hot reservoir and 
\begin{equation}\label{eq:QC}
Q_{41} = - Q_{C} = W_{41} = mRT_C\ln\left(\frac{V_1}{V_4}\right)
\end{equation}
where $Q_C$ refers to the heat transferred from the system to the cold reservoir. 
\subsubsection{Adiabatic Processes}
For adiabatic, reversible processes $1 \to 2$ and $3 \to 4$ the work is given by
\begin{equation}
W_{12} = \int_{V_1}^{V_2} \frac{C_{12}}{V^{\gamma}} dV \text{ and } W_{34} = \int_{V_3}^{V_4} \frac{C_{34}}{V^{\gamma}} dV
\end{equation}
which, when integrated yield
\begin{equation}
\begin{gathered}
W_{12} =\frac{-C_{12}}{\gamma-1}\left(\frac{1}{V_2^{\gamma-1}} - \frac{1}{V_1^{\gamma-1}}\right) \\
\text{ and } \\
 W_{34} = \frac{-C_{34}}{\gamma-1}\left(\frac{1}{V_4^{\gamma-1}} - \frac{1}{V_3^{\gamma-1}}\right) 
\end{gathered}
\end{equation}
Multiplying both sides of the relation in Eqn.~\eqref{eq:polytropic} by $1/V^{\gamma-1}$ and using the resulting relation, $pV = C/V^{\gamma-1}$, combined with the ideal gas equation of state, we can express the work in the adiabatic, reversible processes as
\begin{equation}
W_{12} =\frac{mR(T_C - T_H)}{\gamma - 1}  \,\,\text{ and }\,\, W_{34} = \frac{mR(T_H - T_C)}{\gamma - 1}
\end{equation}
illustrating that the work in these two processes are equal in magnitude but opposite in sign. 
It is interesting to determine the relationship between volume and temperature in the end states of processes $1 \to 2$ and $3 \to 4$ for the ideal gas. For an infinitesimal change in internal energy $dU$ no heat flow occurs, but an infinitesimal amount of $pV$ work is done, $\delta W = p dV$. We write this infinitesimal form of the energy balance as
\begin{equation} 
d U = \cancelto{0}{\delta Q} - \delta W = -p dV
\end{equation}
For an ideal gas, $dU/dT = mc_v$ and $p = mRT/V$ so that
\begin{equation}
c_v dT = -\frac{RT}{V} dV
\end{equation}
Separating the volume $V$ and temperature $T$ and integrating, between states $1$ and $2$ yields
\begin{equation} \label{eq:VT12}
\int_{T_C}^{T_H} \frac{c_v}{RT} dT =  \ln\left(\frac{V_1}{V_2}\right).
\end{equation}
Integrating between states $4$ and $3$ yields 
\begin{equation} \label{eq:VT43}
\int_{T_C}^{T_H} \frac{c_v}{RT} dT = \ln\left(\frac{V_4}{V_3}\right).
\end{equation}
The left hand side of both Eqns.~\eqref{eq:VT12} and \eqref{eq:VT43} are equal so that 
\begin{equation}
\ln\left(\frac{V_1}{V_2}\right) = \ln\left(\frac{V_4}{V_3}\right).
\end{equation}
or equivalently\sidenote{ $\ln(x/y) = \ln x - \ln y$.}
\begin{equation} \label{eq:V1234}
\ln\left(\frac{V_1}{V_4}\right) = \ln\left(\frac{V_2}{V_3}\right).
\end{equation}
Recalling the results from the energy balances on the isothermal processes (Eqns.~\eqref{eq:QH} and \eqref{eq:QC}) we can show that the ratio of the heat influx $Q_H$ to heat loss $Q_C$ is
\begin{equation}
\frac{Q_H}{Q_C} = \frac{T_H\ln\left(\frac{V_3}{V_2}\right) }{T_H\ln\left(\frac{V_4}{V_1}\right)}
\end{equation}
Applying Eqn.~\eqref{eq:V1234}, which defines the relationship between the states before and after the adiabatic processes, we obtain
\begin{equation} \label{eq:HeatTemp}
\frac{Q_H}{Q_C} = \frac{T_H}{T_C}
\end{equation}
for an ideal gas undergoing a Carnot cycle. 
\subsubsection{Internal Energy Balance of the System}
The total change in the internal energy of the Carnot cycle $\Delta U$ can alternately be determined by an energy balance on the cycle
\begin{align}
\Delta U&= Q - W \label{eq:Ebalance} \\
&= Q_H - Q_C - W_{12} - \underbrace{W_{23}}_{=Q_H} - \underbrace{W_{34}}_{=-W_{12}} - \underbrace{W_{41}}_{=-Q_C} \label{eq:Ebaldetailed} \\
&= 0
\end{align}
Thus the Carnot cycle is consistent with the energy change for a cycle in general.
\subsubsection{Thermal Efficiency}
This Carnot cycle turns heat input $Q_H$ into work output $W$. The thermal efficiency $\eta$ for such a \textbf{heat engine} is given by 
\begin{equation}
\eta = \frac{\text{net work output}}{\text{heat input}} = \frac{W}{Q_H}
\end{equation}
Combining $\Delta U = 0$ for a cycle, the expression for the energy balance on the cycle (Eqns.~\eqref{eq:Ebalance} and \eqref{eq:Ebaldetailed}), and the temperature-heat transfer relationship found in Eqn.~\eqref{eq:HeatTemp} yields
\begin{equation}
\eta = \frac{Q_H - Q_C}{Q_H} = 1-\frac{T_C}{T_H}.
\end{equation}
A Carnot cycle produces the maximum possible efficiency for any heat engine. The efficiency can only approach $1$ (equivalent to 100\%) when either $T_C \to 0$ or $T_H \to \infty$, neither of which is physically attainable. 
%%%%%%%%%%%%%%%%%%%%%%%%%%%%%%%%%%%%%%%%%%%%%%%%%%%%%%%%%%%%%%%%%%%%%%%%%%%%%%%%%%%%%%%%%%%%%%%%%%%%%%%%%%%%%%%%%%%%%%%%%%%%%%%%%%%
\subsection{A Rankine Cycle}
The Rankine cycle is the basis for steam-electric power plants, which produce nearly 90\% of all electricity worldwide.  This cycle includes two isentropic processes, and utilizes isobaric heat transfers. 
\begin{figure}[h]
\begin{center}
\begin{tikzpicture}[auto,every text node part/.style={align=center}]

	\node (hexIN) [draw, fill = Dorange, rectangle, minimum width = 2cm, minimum height = 3cm, very thick]{};
	\node (hexLabel) at (hexIN) [draw=none,color=black,xshift=-0.4cm] {\small{heat}\\[-5pt]\small{source}\\ \tiny{coal, sun,}\\[-8pt] \tiny{nuclear, etc.}};
	\node (turbine) at ($(hexIN) + (2,4)$) [draw, very thick, trapezium,trapezium left angle=70,trapezium right angle=70,rotate=90,minimum width=0.75cm, minimum height=0.75cm, 
	text width=1cm, text height = 1cm, text depth = 0cm, text centered,thick,yshift=-1cm, xshift=-2cm] {};
	\node (turbineLabel) at (turbine) [draw=none] {\small{turbine}};
	\node (hexOUT) at ($(turbine) + (0.65,-3)$) [draw, fill=Dteal!20,rectangle, minimum width = 1.5cm, minimum height = 2cm, very thick]{};
	\node (pump) at ($(hexIN) + (1.5,-3)$) [draw, circle, minimum width=0.75cm, very thick] {\small{pump}};
	
	
	\node (electric) at ($(turbine) + (2,0)$) [draw,rectangle, minimum width = 1.5cm, minimum height = 1cm, very thick, rounded corners] {\small{electric} \\[-5pt] \small{generator}};
	
	% power
	\draw [line width=10pt] (turbine.south) -- (electric.west);
	\draw [thick,latex-,color=Dlblue] ($(pump)+(0,-0.25)$) -- ++(1,-0.6) node[right] {$\dot{W}_p$};
	\draw [thick,-latex,color=Dlblue] (electric.east) -- ++(1,0) node[right] {$\dot{W}_{\text{out}}$};	

	% working fluid flow
	\draw [-,ultra thick,double, decorate,decoration={snake,amplitude=3mm,segment length=5mm, post length=8pt, pre length=10pt}] ($(hexIN.south)+(0.5,0)$) node[above,xshift=-0.25cm]{}-- ++(0,3);
	\draw [-latex,ultra thick,double] ($(hexIN.north)+(0.5,0)$) |- (turbine.north)  node[above, xshift=-0.5cm]{\circled{1}};
	\draw [-latex,ultra thick,double] (turbine.200) -- ($(hexOUT.north)+(-0.3,0)$) node[above, xshift=0.5cm]{\circled{2}};
	\draw [-,ultra thick,double, decorate,decoration={snake,amplitude=2.5mm,segment length=4mm, post length=4pt, pre length=5pt}] ($(hexOUT.north)+(-0.3,0)$) node[above,xshift=-0.25cm]{}-- ++(0,-2);
	\draw [-latex,ultra thick,double] ($(hexOUT.south)+(-0.3,0)$) node[right, yshift=-0.5cm]{\circled{3}} |- (pump.east) ;
	\draw [-latex,ultra thick,double] (pump.west) -| ($(hexIN.south)+(0.5,0)$) node[above, xshift=-0.5cm, yshift=-1.5cm]{\circled{4}};
	
	% cooling fluid
	\draw [-latex, ultra thick, double,color=Dteal] ($(hexOUT.east)+(1,-0.75)$) node[right] {\small{from cooling}\\[-5pt] \small{tower}}-- ($(hexOUT.east)+(-0.35,-0.75)$) -- ($(hexOUT.east)+(-0.35,0.75)$) -- ($(hexOUT.east)+(1,0.75)$) node[right] {\small{to cooling}\\[-5pt] \small{tower}};


\end{tikzpicture}\end{center}
\caption{A schematic of a vapor power plant capable of undergoing a Rankine cycle.}\label{fig:IdealRankineCycle}
\end{figure}

An ideal Rankine cycle consists of the following internally reversible processes: 
\begin{itemize}
\item \textbf{Process $1\to 2$:} Isentropic expansion of the working fluid through the turbine from saturated vapor at state $1$ to the condenser pressure $p_2$.
\item \textbf{Process $2\to 3$:} Heat transfer from the working fluid as it flows at constant pressure $p_2 = p_3$ through the condenser, exiting as a saturated liquid at state $3$. 
\item \textbf{Process $3\to 4$:} Isentropic compression in the pump to state $4$ in the compressed liquid region of the phase diagram.
\item \textbf{Process $4\to 1$:} Heat transfer to the working fluid as it flows at constant pressure $p_4 = p_1$ through the boiler to complete the cycle. 
\end{itemize}


Instead, the condenser stream is taken to the saturated liquid state, so that only a compressed liquid must be pumped.\sidenote{Cavitation is less of an issue in the two phase fluid used in the turbine since the vapor bubbles that form nearer the saturated vapor portion of the vapor dome are less driven to collapse.} While state $4$ could be taken all the way to $T_{\text{sat}}(p_4 = p_1)$, this would also lead to practical difficulties since, to maintain an isothermal heat transfer, a large thermal reservoir would be required. Instead, the process from state $4$ to state $1$ follows an isobar as shown in the plot below. 

\begin{figure}
\begin{tikzpicture}[auto,>=latex',
	declare function={
        		curveL(\x) = -1.6*\x^2-0.25*x^3+10;
		curveR(\x) = -1*\x^2 + 0.3*x^2.5 + 10;
  	}
	]

	%%% Example isobar
	
	\node (ref) at (0,0) [draw=none, coordinate] {};
	\begin{axis} [at={(ref)},xmin=-4, ymin=0, xmax=7, ymax = 11, samples = 50, 
			ytick={2,8}, yticklabels={$T_{\text{sat}}(p_2 = p_3)$,$T_{\text{sat}}(p_4 = p_1)$},  ylabel style={rotate=-90},
			xtick=\empty, xticklabels=\empty,
			ylabel={$T$}, xlabel={$s$},
			width = 6cm, height=6cm, axis y line = left, axis x line = bottom,
			thick,ylabel style = {at={(axis description cs:0,0.95)},anchor=south east}, xlabel style = {at={(axis description cs:0.95,0)},anchor=north east}]

    	%L+V dome
            		\addplot[thick,domain=-5:0, name path=L] {curveL(x)};	
            		\path[name path=axL] (axis cs:-5, 0) -- (axis cs:0, 0);	
            		\addplot[thick,domain=0:7, name path=R] {curveR(x)};
            		\path[name path=axR] (axis cs:0, 0) -- (axis cs:14, 0);
	%isobar
			\addplot[ultra thick, mark=none,color=Dlorange] coordinates {(-4,3.5) (-1.25,8) (1.84,8) (3,10.5)} node[right,xshift=0.05cm]{isobar};
			
	% Tavg
			\addplot[thick,  dashed,color=Dteal] coordinates {(-4,7.1) (5,7.1)};
			\node [above, color=Dteal] at (axis cs:5, 7) {$T_{\text{avg}}$};
			
				% Cycle  		
			\addplot[thick,  ->] coordinates {(-1.25,8) (1.84,8)};
			\addplot[thick, ->] coordinates {(1.84,2) (-3.13,2)};
			\addplot[ thick,->] coordinates {(-3.13,2) (-3.13,5)};
			\addplot[thick] coordinates {(-3.13,5) (-1.25,8)};
			\addplot[thick, ->] coordinates {(1.84,8) (1.84,2)};
			\addplot [only marks, thick] coordinates {
			(-3.13,2) (-3.13,5)
			(1.84,2) (1.84,8)
			};

			
		\node [below, color=black] at (axis cs:-3.13, 2) {\circled{3}};
		\node [above, color=black] at (axis cs:-3.13, 5) {\circled{4}};
		\node [right, color=black] at (axis cs:1.84, 8) {\circled{1}};		
		\node [right, color=black] at (axis cs:1.84, 2) {\circled{2}};

	\end{axis}

\end{tikzpicture}
\caption{A $T$-$s$ diagram of an ideal Rankine cycle.}\label{fig:TsDiagramRankine}
\end{figure}

For reversible processes, the second law can be written as:
\begin{equation}
Q = \int T dS,
\end{equation}
 meaning that the area under the curve in a $T$-$s$ diagram corresponds to heat transferred to or from a system. Thus to approximate a Carnot cycle, an average temperature value at which the same heat transfer $Q_H$ occurs can be approximated on the diagram.
Since the Carnot cycle tells us that the maximum thermal efficiency of a power cycle is:
\begin{equation}
\eta = 1- Q_C/Q_H = 1-T_C/T_H
\end{equation} 
it is clear from Fig.~\ref{fig:TsDiagramRankine} that a hypothetical Carnot cycle with maximum operating temperature $T_H$ will always have a greater thermal efficiency than a Rankine cycle with the same maximum operating temperature since the \emph{average} operating temperature of the Rankine cycle is lower. 


\subsection{Otto Cycle}
The Otto cycle provides an approximation of the internal combustion engines that still make up the largest share of the transportation industry. 

An automotive internal combustion engine uses a reciprocating piston-cylinder action to produce work.  In a four-stroke engine
\begin{enumerate}
\item The intake valve is open and the piston makes an intake stroke to draw in a fresh charge, e.g., a combustible mixture of fuel and air. 
\item The intake valve closes and the piston undergoes a compression stroke that increases the pressure and temperature of the air within. At the end of the compression, combustion is induced, e.g., via a spark plug.
\item A power stroke follows as the gas mixture expands, doing work on the piston.
\item Finally, the exhaust valve is opened and burned gases are purged during an exhaust stroke.
\end{enumerate}

The Otto cycle simplifies this process by ignoring affects associated with the addition and removal of fuel. Instead it uses air, acting as an ideal gas, to provide insight into how such cycles can be optimized. The combustion itself is replaced by heat transfer and all processes are internally reversible. As a result, the Otto cycle consists of the following steps.
\begin{itemize}
\item \textbf{Process 1-2:} Isentropic compression of the air as the piston moves from bottom to top.
\item \textbf{Process 2-3:} Constant-volume heat transfer to the air from an external source while the piston is at the top. (Approximates fuel ignition and rapid burning.)
\item \textbf{Process 3-4:} Isentropic expansion (power stroke).
\item \textbf{Process 4-1:} Constant-volume process in which heat is rejected from the air while the piston is at the bottom.
\end{itemize}
Note that because it is operating within a piston cylinder, the states the processes operate between are the state of the air in the closed piston cylinder system. This is different from the Rankine cycle, in which the stream states change by going through processes facilitated by different thermodynamics devices. We will show in class that the thermal efficiency of an Otto cycle can be expressed entirely as a function of the compression ratio, $r$, defined as the ratio of the largest gas volume over the smallest $V_1/V_2$ according to:
\begin{equation}
\eta = 1-\frac{1}{r^{k-1}}
\end{equation}
and shows that engine efficiency can be increased by inducing a larger change in piston chamber volume. Of course, further practical considerations, such as the increased weight of such an engine, provide other design constraints. We only touch on a few of these design constraints, but in general thermodynamics provides many of the key principles through which energy systems can be optimized. 

